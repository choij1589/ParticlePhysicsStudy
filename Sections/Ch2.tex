\noindent\rule{7in}{2.8pt}
\section{Underlying Concepts}
\begin{problem}{2.1}
When expressed in natural units the lifetime of the W boson is approximately $\tau \approx 0.5\GeV^{-1}$. What is the corresponding value in S.I. units?
\end{problem}

\begin{solution}

\end{solution} 
\noindent\rule{7in}{1.5pt}

%%%%%%%%%%%%%%%%%%%%%%%%%%%%%%%%%%%%%%%%%%%%%%%%%%%%%%%%%%%%%%%%%%%%%%%%%
% Problem 2
%%%%%%%%%%%%%%%%%%%%%%%%%%%%%%%%%%%%%%%%%%%%%%%%%%%%%%%%%%%%%%%%%%%%%%%%%%%%%%%%%%%%%%%%%%%%%%%%%%%%%%%%%%%%%%%%%%%%%%%%%%%%%%%%%%%%%%%%

\begin{problem}{2.2}
A cross section is measured to be 1 pb; convert this to natural units.
\end{problem}

\begin{solution}
Taking note that $\hbar c = 0.197 \GeV\cdot\text{fm}$
\begin{align*}
    1 \pb = 100 \text{ fm}^2 = 100 \times \left( \frac{\hbar c }{0.197} \right)^2 \GeV^{-2} = 100 \times \left( \frac{\num{1.055e-34}\times\num{2.998e8}}{0.197} \right) \GeV^{-2}
\end{align*}
\end{solution} 
\noindent\rule{7in}{1.5pt}

%%%%%%%%%%%%%%%%%%%%%%%%%%%%%%%%%%%%%%%%%%%%%%%%%%%%%%%%%%%%%%%%%%%%%%%%%

\begin{problem}{2.3}
Show that the process $\gamma\to e^+ e^-$ can not occur in vacuum.
\end{problem}

\begin{solution}

\end{solution} 
\noindent\rule{7in}{1.5pt}

%%%%%%%%%%%%%%%%%%%%%%%%%%%%%%%%%%%%%%%%%%%%%%%%%%%%%%%%%%%%%%%%%%%%%%%%%

\begin{problem}{2.4}
A particle of mass 3 GeV is travelling in the positive z-direction with momentum 4 GeV. What are its energy and velocity?
\end{problem}

\begin{solution}
Using the relation of $m^2=E^2- | \boldp |^2 $, one gets $E^2 = 25 \GeV^2$ thus the energy is $E=5\GeV$. Now considering the relation of $|\boldp|=E\beta$, it is seen that $\beta = |\boldp|E^{-1}=0.8$ thus the velocity is $0.8c$.\\ 
\end{solution} 
\noindent\rule{7in}{1.5pt}

%%%%%%%%%%%%%%%%%%%%%%%%%%%%%%%%%%%%%%%%%%%%%%%%%%%%%%%%%%%%%%%%%%%%%%%%%

\begin{problem}{2.5}
In the laboratory frame, denoted $\Sigma$, a particle travelling in the z-direction has momentum $\mathbf{p}=p_z \hat{\mathbf
z}$ and energy $E$.
\begin{enumerate}[label=(\alph*)]
    \item Use the Lorentz transformation to find expressions for the momentum $p'_z$ and energy $E'$ of the particle in a frame $\Sigma'$ which is moving in a velopcity $\mathbf{v}=+v\hat{\mathbf
    z}$ relative to $\Sigma$, and show that $E^2-p_z^2=(E')^2-(p'_z)^2$.
    \item For a system of particles, prove that the total four-momentum squared, 
    \begin{align*}
        p^\mu p_\mu \equiv  \left( \sum_{i}E_i \right)^2 - \left( \sum_{i}\mathbf{p}_i \right)^2
    \end{align*}
    is invariant under Lorentz transformations.
\end{enumerate}
\end{problem}

\begin{solution}
    \begin{enumerate}[label=(\alph*)]   
        \item Let the four-momentum of the given particle in the frame $\Sigma$ and $\Sigma'$ as $p=(E,0,0,p_z),p'=(E',\boldp')$ respectively. Denoting the corresponding matrix representation of the given Lorentz transformation as $\boldsymbol{\Lambda}$, one could write down the transformation of $p$ as,
        \begin{align*}
            p' = \boldsymbol{\Lambda} p \implies p'^\mu &= \Lambda^{\mu}_\nu p^\nu \\[0.1in]
                   &= \begin{pmatrix}
                    \gamma       & 0 & 0 & -\gamma\beta \\
                    0            & 1 & 0 & 0            \\
                    0            & 0 & 1 & 0            \\
                    -\gamma\beta & 0 & 0 & \gamma      
                   \end{pmatrix}
                   \begin{pmatrix}
                    E \\
                    0 \\
                    0 \\
                    p_z 
                   \end{pmatrix} = \gamma
                   \begin{pmatrix}
                    E-\beta p_z \\ 0 \\ 0 \\ -E\beta +  p_z \\
                   \end{pmatrix} 
        \end{align*}\\
        which implies that $E' = \gamma \left( E-\beta p_z \right)$ and $p_z'=-\gamma(E\beta-p_z)$. Using such expression of $p'$, one could show that : 
        \begin{align*}
            \left(E'\right)^2 - \left(p_z'\right)^2 &= \gamma^2 (E-\beta p_z)^2 - \gamma^2 (E\beta - p_z)^2 \\[0.1in]
                                                    &= \gamma^2 \left[  \left( E-\beta p_z \right)^2 - \left(E\beta - p_z \right)^2 \right] \\[0.1in]
                                                    &= \gamma^2 \left[  \left( E-\beta p_z + E\beta - p_z \right) \left( E-\beta p_z - E\beta + p_z  \right) \right] \\[0.1in]
                                                    &= \gamma^2 (1+\beta)(1-\beta) (E-p_z)(E+p_z) = E^2 - p_z^2 \qed
        \end{align*}
        \item 
    \end{enumerate}
\end{solution} 
\noindent\rule{7in}{1.5pt}

%%%%%%%%%%%%%%%%%%%%%%%%%%%%%%%%%%%%%%%%%%%%%%%%%%%%%%%%%%%%%%%%%%%%%%%%%

\begin{problem}{2.6}\label{p2.6}
   For the decay $a\to1+2$, show that the mass of the particle $a$ can be expressed as 
   \begin{align*}
        m_a^2 = m_1^2+m_2^2 + 2E_1E_2\left( 1 - \beta_1\beta_2 \cos\theta \right)
   \end{align*}
   where $\beta_1$ and $\beta_2$ are the velocities of the daughter particles and $\theta$ is the angle between them.
\end{problem}
    
\begin{solution}
    Let the four-momenta of the daughters as $p_i = \left( E_i,\boldp_i \right)$ for $i=1,2$. Momentum conservation states that $p_a = p_1 + p_2$ where $p_a$ is the four-momentum of the mother particle.
    Squaring both sides, one obtains 
    \begin{align*}
        p_a \cdot p_a = m_a^2 &= \left( p_1 + p_2 \right)^2 \\[0.1in]
                              &= p_1 \cdot p_1 + p_2 \cdot p_2 + 2 p_1 \cdot p_2 \\[0.1in]
                              &= m_1^2 + m_2^2 + 2 (E_1E_2 - \boldp_1 \cdot \boldp_2) \\[0.1in]
                              &= m_1^2 + m_2^2 + 2 (E_1E_2 - |\boldp_1||\boldp_2|\cos\theta  ) \\[0.1in]
                              &= m_1^2 + m_2^2 + 2 (E_1E_2 - E_1\beta_1E_2\beta_2\cos\theta  ) \\[0.1in]
                              &= m_1^2+m_2^2 + 2E_1E_2\left( 1 - \beta_1\beta_2 \cos\theta \right) \qed 
    \end{align*}
\end{solution} 
    
\noindent\rule{7in}{1.5pt}

%%%%%%%%%%%%%%%%%%%%%%%%%%%%%%%%%%%%%%%%%%%%%%%%%%%%%%%%%%%%%%%%%%%%%%%%%

\begin{problem}{2.7}
In a collider experiment, $\Lambda$ baryons can be identified from the decay $\Lambda\to\pi^-p$, which gives rise to a displaced vertex in a tracking detector. In a particular decay, the momenta of the $\pi^+$ and $p$ are measured to be $0.75 \GeV$ and $4.25 \GeV$ respectively, and the opening angle between the tracks is $9^\circ$. The masses of the pion and proton are $189.6 \MeV$ and $938.3 \MeV$.
\begin{enumerate}[label=(\alph*)]
    \item Calculate the mass of the $\Lambda$ baryon.
    \item On average, $\Lambda$ baryons of this energy are observed to decay at a distance of 0.35 m from the point of production. Calculate the lifetime of the $\Lambda$.
\end{enumerate}
\end{problem}

\begin{solution}

\end{solution} 
\noindent\rule{7in}{1.5pt}


%%%%%%%%%%%%%%%%%%%%%%%%%%%%%%%%%%%%%%%%%%%%%%%%%%%%%%%%%%%%%%%%%%%%%%%%%

\begin{problem}{2.8}
    In the laboratory frame, a proton with total energy E collides with proton at rest. Find the minimum proton energy such that process
    \begin{align*}
        p+p \to p+p+\bar{p}+\bar{p}
    \end{align*}
    is kinematically allowed.
\end{problem}

\begin{solution}
        
\end{solution} 

\noindent\rule{7in}{1.5pt}

%%%%%%%%%%%%%%%%%%%%%%%%%%%%%%%%%%%%%%%%%%%%%%%%%%%%%%%%%%%%%%%%%%%%%%%%%

\begin{problem}{2.9}
Find the maximum opening angle between the photons produced in the decay $\pi^0\to\gamma\gamma$ if the energy of the neutral pion is 10 GeV, given that $m_{\pi^0}=135$ MeV.
\end{problem}

\begin{solution}
Using the results derived in Problem~\ref{p2.6} and taking account on the fact that photons are massless, one could write down 
\begin{align*}
    m_{\pi_0^2} = 2E_1E_2 \left( 1- \beta_1 \beta_2 \cos \theta \right) = 2E_1E_2 \left( 1- \cos\theta \right) \implies \cos \theta =  \frac{m_{\pi_0}^2}{2E_1E_2} - 1  
\end{align*}
Taking account that $E_1+E_2 = 10 \GeV$, let $E_1=E$ and express $\theta$ in terms of $E$ as,
\begin{align*}
    \cos \theta =  \frac{m_{\pi_0}^2}{2E(10-E)} - 1  
\end{align*}
In the range of $E\in[0,10]\GeV$ the RHS of the above identity will take a local minimum when $E=5\GeV$ which will give the maximum value of $\theta$, which will be denoted as $\theta^\ast$. One could get $\theta^\ast$ as,
\begin{align*}
    \cos \theta^\ast = \frac{\left( \num{1.35e-1}\GeV \right)^2}{100 \GeV^2} - 1 = 
\end{align*}
\end{solution} 

\noindent\rule{7in}{1.5pt}

%%%%%%%%%%%%%%%%%%%%%%%%%%%%%%%%%%%%%%%%%%%%%%%%%%%%%%%%%%%%%%%%%%%%%%%%%

\begin{problem}{2.10}
The maximum of the $\pi^-p$ cross section, which occurs at $p_\pi=300$ MeV, corresponds to the resonant production of the $\Delta^0$ baryon (i.e. $\sqrt{s}=m_\Delta$). What is the mass of the $\Delta$?
\end{problem}

\begin{solution}
        
\end{solution} 

\noindent\rule{7in}{1.5pt}

%%%%%%%%%%%%%%%%%%%%%%%%%%%%%%%%%%%%%%%%%%%%%%%%%%%%%%%%%%%%%%%%%%%%%%%%%

\begin{problem}{2.11}
   Tau-leptons are produced in the process $e^+e^-\to\tau^+\tau^-$ at a centre-of-mass energy of 91.2 GeV. The angular distribution of the $\pi^-$ from the decay $\tau^-\to\pi^-\nu_\tau$ is 
   \begin{align*}
        \frac{\dif N}{ \dif \left(\cos\theta^\ast\right)} \varpropto 1 + \cos\theta^\ast
   \end{align*} 
   where $\theta^\ast$ is the polar angle of the $\pi^-$ in the tau-lepton rest frame, relative to the direction defined by the $\tau$ spin. Determine the laboratory frame energy distribution of the $\pi^-$ for the cases where the tau lepton spin is (i) \textit{aligned with} or (ii) \textit{opposite} to its direction of flight.
\end{problem}
    
\begin{solution}
            
\end{solution} 
    
\noindent\rule{7in}{1.5pt}

%%%%%%%%%%%%%%%%%%%%%%%%%%%%%%%%%%%%%%%%%%%%%%%%%%%%%%%%%%%%%%%%%%%%%%%%%


\begin{problem}{2.12}
   For the process $1+2 \to 3+4$, the Mandelstam variables $s,t$ and $u$ are defined as $s=(p_1+p_2)^2,t=(p_1-p_3)^2$ and $u=(p_1-p_4)^2$. Show that
   \begin{align*}
        s+t+u = m_1^2+m_2^2+m_3^2+m_4^2.
   \end{align*} 
\end{problem}
    
\begin{solution}
    By definition of the Mandelstam variables, one could express $(s+t+u)$ as 
    \begin{align*}
        s+t+u &= (p_1+p_2)^2 + (p_1-p_3)^2 + (p_1-p_4)^2\\[0.1in]
              &= \sum_i p_i \cdot p_i  + 2p_1\cdot p_1  + 2p_1\cdot p_2 -2 p_1 \cdot p_3 - 2 p_1 \cdot p_4  \\  
              &= \sum_i m_i^2 + 2p_1\cdot(p_1+p_2-p_3-p_4) \\
              &= \sum_i m_i^2 \qed
    \end{align*}
    The fact that in any frame $p^\mu p_\mu =m^2$ for a particle with mass $m$ is used in the third identity, and in the last step the conservation of momentum $p_1+p_2 = p_3+p_4$ is used. 
\end{solution} 
    
\noindent\rule{7in}{1.5pt}

%%%%%%%%%%%%%%%%%%%%%%%%%%%%%%%%%%%%%%%%%%%%%%%%%%%%%%%%%%%%%%%%%%%%%%%%%

\begin{problem}{2.13}
 At the HERA collider, 27.5 GeV electrons were collided head-on with 820 GeV protons. Calculate the centre-of-mass energy.
\end{problem}
    
\begin{solution}
    Let the four-momentum of the electron and proton as $p_e = (E_e,\mathbf{p}_e),$ $p_p = (E_p,\mathbf{p}_p)$ respectively. The centre-of-mass energy $\sqrt{s}$ can be expressed as,
    \begin{align*}
        s &= \left( p_e + p_p \right)^2 = p_e\cdot p_e + p_p \cdot p_p + 2 p_e \cdot p_p \\[0.1in]
          &= m_e^2 + m_p^2 + 2 \left( E_e E_p - \boldp_e \cdot \boldp_p \right) \\[0.1in]
          &= m_e^2 + m_p^2 + 2 \left( E_eE_p + \left| \boldp_e \right| \left| \boldp_p \right| \right) \simeq 4 E_e E_p \quad \quad \left(  |\boldp_i|^2 = E_i^2 - m_i^2 \sim E_i^2  \right)
    \end{align*}
    As the collision is occuring head-on, one could say that $\boldp_e \cdot \boldp_p =- \left| \boldp_e \right| \left| \boldp_p \right| $ which was used in the last identity.
    Looking upon the order of the variables, $m_e \simeq 0.5 \MeV, m_p \simeq 93.8 \MeV$ and $E_e = 27.5 \GeV, E_p = 820 \GeV$ for an approximation it is okay to consider $m_e,m_p\sim 0$. Thus the centre-of-mass energy $\sqrt{s}\simeq 300 \GeV$ when all the needed values are plugged in.
\end{solution}
    
\noindent\rule{7in}{1.5pt}

%%%%%%%%%%%%%%%%%%%%%%%%%%%%%%%%%%%%%%%%%%%%%%%%%%%%%%%%%%%%%%%%%%%%%%%%%

\begin{problem}{2.14}

\end{problem}
    
\begin{solution}
            
\end{solution}
    
\noindent\rule{7in}{1.5pt}

%%%%%%%%%%%%%%%%%%%%%%%%%%%%%%%%%%%%%%%%%%%%%%%%%%%%%%%%%%%%%%%%%%%%%%%%%

\begin{problem}{2.15}

\end{problem}
    
\begin{solution}
            
\end{solution}
    
\noindent\rule{7in}{1.5pt}

%%%%%%%%%%%%%%%%%%%%%%%%%%%%%%%%%%%%%%%%%%%%%%%%%%%%%%%%%%%%%%%%%%%%%%%%%

\begin{problem}{2.16}

\end{problem}
    
\begin{solution}
            
\end{solution}
    
\noindent\rule{7in}{1.5pt}



%%%%%%%%%%%%%%%%%%%%%%%%%%%%%%%%%%%%%%%%%%%%%%%%%%%%%%%%%%%%%%%%%%%%%%%%%

\begin{problem}{2.17}
Find the third-order term in the transition matrix element of Fermi's golden rule.
\end{problem}

\begin{solution}
        
\end{solution} 

\noindent\rule{7in}{1.5pt}